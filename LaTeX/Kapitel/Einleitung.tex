\chapter{Einleitung}
\label{Chapter::Einleitung}

In der modernen Datenanalyse spielen gemischte Modelle eine zentrale Rolle, da sie es ermöglichen, sowohl feste als auch zufällige Effekte zu berücksichtigen, was sie besonders in den Bereichen der Biostatistik, der Sozialwissenschaften und der ökonomischen Modellierung beliebt macht. Mit dem Aufkommen von Big Data und komplexen Datenstrukturen hat sich der Fokus zunehmend auf die effiziente und genaue Extraktion von Informationen aus großen und oft unübersichtlichen Datensätzen verschoben. In diesem Kontext gewinnen latente Repräsentationen an Bedeutung, da sie es ermöglichen, die inhärenten Strukturen innerhalb der Daten zu identifizieren und zu nutzen, um tiefergehende Einsichten zu gewinnen.

Jedoch birgt die Integration von gemischten Modellen in latente Repräsentationen das Risiko einer Verzerrung der Inferenzergebnisse, was die Genauigkeit und Zuverlässigkeit der aus den Daten gezogenen Schlussfolgerungen erheblich beeinträchtigen kann. Diese Arbeit beschäftigt sich daher mit der Untersuchung der Verzerrungen, die bei der Anwendung gemischter Modelle auf latente Repräsentationen auftreten können. Ziel ist es, die Mechanismen zu verstehen, die zu diesen Verzerrungen führen, und Methoden zu entwickeln, um ihre Auswirkungen zu minimieren.

Die Fragestellung der Verzerrung ist besonders relevant, da eine fehlerhafte Inferenz zu falschen Entscheidungen führen kann, die in praktischen Anwendungen schwerwiegende Folgen haben könnten. Durch eine sorgfältige Analyse und Evaluation der gemischten Modellansätze in Verbindung mit latenten Repräsentationen strebt diese Arbeit an, einen Beitrag zur Verbesserung der Modellgenauigkeit und der Verlässlichkeit von Inferenzschlüssen zu leisten.

Diese Arbeit gliedert sich in mehrere Teile, die zunächst die theoretischen Grundlagen der gemischten Modelle und der latenten Repräsentationen behandeln, gefolgt von einer Diskussion der Methoden zur Messung und Korrektur von Verzerrungen. Anhand von experimentellen Studien werden diese Konzepte dann praktisch angewendet und evaluiert, um abschließend Empfehlungen für die Anwendung dieser Techniken in der Forschung und Praxis zu geben.
\section{Motivation}
\label{Section::Motivation}
  
\section{Problem and Research Questions}
\label{Section::Problem_and_Research_Questions}

%\ifstandalone
%    \printglossary
%    \printbibliography[heading=bibintoc]
%\fi

\end{document}