% Formatvorlage für Abschlussarbeiten an der Universität Wien
%
% Diese Vorlage ist für die Einreichung von Masterarbeiten vorgesehen und basiert auf einer Vorlage von Martin Dockner. Sie entspricht den verbindlichen Vorgaben zur Titelblattgestaltung an der Universität Wien. 
%
% Fügen Sie in diese Vorlage mittels Copy & Paste die Daten ein, die Ihnen im Schritt 6 im Prozess zum Upload der Abschlussarbeit zur Plagiatsprüfung in u:space angezeigt werden. Verändern Sie die Daten bitte nicht. Ein fehlerhaftes Titelblatt kann das Abschlussprozedere verzögern.  
%
% Version 0.9 2023
%
% Diese Vorlage nutzt die KOMA-Script-Klassen. Für Details siehe hier:
% ftp://ftp.dante.de/pub/tex/macros/latex/contrib/koma-script/scrguide.pdf
%%%%%%%%%%%%%%%%%%%%%%%%%%%%%%%%%%%%%%%%%%%%%%%%%%%%%%%%%%%%%%%%%%%%%%%%%%%%%%

\documentclass[12pt,pagesize,twoside,DIV=12,headsepline=0.4pt]{scrbook}

% Verwenden des Fonts Source Sans Pro
\usepackage[default,semibold,light]{sourcesanspro}
\usepackage[T1]{fontenc}


% \usepackage[rm={lining,proportional},sf={lining,proportional},tt={lining,tabular,monowidth}]{cfr-lm}
% \global\let\bfseries\sbweight



% folgende drei Pakete sind nur in deutsch wichtig! Sie können für englische
% Arbeiten auskommentiert werden ( % davor setzen)
\usepackage[ngerman]{babel}		% deutsche Abteilungsregeln
\usepackage{lmodern} 	        % notwendig damit im PDF nach Umlauten gesucht werden kann
\usepackage[utf8]{inputenc}     % notwendig für Darstellung von Umlauten und scharfem S
\usepackage[T1]{fontenc}

% weitere wichtige Pakete
\usepackage{graphicx} 			% für Abbildungen benötigt
%\usepackage{natbib,bibgerm} 	% werden für die meisten Literaturstile benötigt
\usepackage[natbibapa]{apacite} % für Literaturestile nach den APA Zitierregeln
\usepackage{booktabs}			% für hübschere Tabellen
\usepackage{lineno}				% für Zeilnennummerierung
\usepackage{setspace}			% für 1.5-fachen oder doppelten Zeilenabstand
%\usepackage[DIV=15]{textarea}   % für Satzspiegeleinstellung/Layout

\usepackage{scrlayer-scrpage}   % für Modifkationen an Kopf- und Fusszeilen
    \pagestyle{scrheadings}     % völlig frei definierte Kopf- und Fusszeile
        \automark[chapter]{chapter} % nötig für den folgenden Befehl
        \ohead[]{\headmark}     % fügt in alle Kopfzeilen den Kapitelnamen ein
        \cfoot[]{}              % die Mitte der Fusszeile bleibt leer!
        \ofoot[\pagemark]{\pagemark} % fügt Seitenzahlen in der Fusszeile ein

% folgende Zeilen aktivieren Serifenschrift im kompletten Dokument
\setkomafont{disposition}{\normalcolor\bfseries}		% für Überschriften
\setkomafont{descriptionlabel}{\normalcolor\bfseries}	% für Descriptions
\setkomafont{captionlabel}{\normalcolor\bfseries}		% für Captions


%%%%%%%%%%%%%%%%%%%%%%%%%%%%%%%%%%%%%%%%%%%%%%%%%%%%%%%%%%%%%%%%%%%%%%%%%%%%%%
% Definition der Persönlichen Daten %%%%%%%%%%%%%%%%%%%%%%%%%%%%%%%%%%%%%%%%%%
%%%%%%%%%%%%%%%%%%%%%%%%%%%%%%%%%%%%%%%%%%%%%%%%%%%%%%%%%%%%%%%%%%%%%%%%%%%%%%
% Persönliche Angaben zur Arbeit folgen hier

\renewcommand{\title}{Zusatz zum Titel Titel meiner Arbeit} % Titel der Arbeit
\newcommand{\akademischegrade}{}            % bereits erlangte akademische Grade!
\renewcommand{\author}{Mag. Max Mustermann, BA}      % Titel Vor Vorname Nachname, Titel Nach
\newcommand{\matrikelnummer}{00000000}      % Matrikelnummer
\newcommand{\studienkennzahl}{A 000 000}                % Studienkennzahl
\newcommand{\studienrichtung}{Masterstudium Beispiel}    % Studienrichtung
\newcommand{\betreuerin}{ao. Univ. Prof. Dr. Max Mustermann, MSc}        % BetreuerIn

% Wenn du das Datum ändern willst, entferne in der nächsten Zeile
% Das Prozentzeichen und paße das Jahr händisch an.
% \year=2013
%%%%%%%%%%%%%%%%%%%%%%%%%%%%%%%%%%%%%%%%%%%%%%%%%%%%%%%%%%%%%%%%%%%%%%%%%%%%%%

\begin{document}						% Hier fängt der Text an
%%%%%%%%%%%%%%%%%%%%%%%%%%%%%%%%%%%%%%%%%%%%%%%%%%%%%%%%%%%%%%%%%%%%%%%%%%%%%%
% Hier wird die Titelseite eingefügt. Sie befindet sich in der Datei
% "titlepage.tex" und entspricht den Vorgaben der Uni Wien.
% Fügen Sie in diese Vorlage mittels Copy & Paste die Daten ein, die Ihnen im Schritt 6 im Prozess zum Upload der Abschlussarbeit zur Plagiatsprüfung in u:space angezeigt werden. Verändern Sie die Daten bitte nicht.
\thispagestyle{empty}
\begin{flushright}
\advance\rightskip-1.5cm\includegraphics[width=7.5cm]{univie}
\end{flushright}

\vspace{0.5cm}
\begin{center}

\huge

\textbf{MASTERARBEIT | MASTER'S THESIS}
%\textbf{DISSERTATION | DOCTORAL THESIS}
\vspace{1.7cm}

\normalsize 
Titel | Title
%Titel der Dissertation
\vspace{0cm}

\Large \title
\vspace{1.7cm}

\normalsize
verfasst von | submitted by
%Titel der Dissertation
\vspace{0cm}

\large \akademischegrade\ \author
\vspace{1.7cm}

\normalsize 
angestrebter akademischer Grad | in partial fulfilment of the requirements for the degree of
\vspace{0cm}

\large 
Master Zusatz (Abkürzung)
%Doktor|in der Beispielwissenschaften (weibliche oder männliche Form)

\vspace{1cm}
\end{center}

\begin{flushleft}
\normalsize
Wien, | Vienna, \the\year \\
\vspace{1cm}

\begin{tabular}{p{8cm}l}
\raggedright Studienkennzahl lt. Studienblatt | degree programme code as it appears on the student record sheet: 	                    & \studienkennzahl	    \\
\raggedright Studienrichtung  lt. Studienblatt | degree programme as it appears on the student record sheet:                       & \studienrichtung      \\
%\raggedright Dissertationsgebiet  lt. Studienblatt | degree programme as it appears on the student record sheet:                       & \studienrichtung      \\
Betreut von | Supervisor:           & \betreuerin			\\
\end{tabular}

\end{flushleft}

%\newpage
%\thispagestyle{empty}
%\mbox{}
% Ende der Titelseite
%%%%%%%%%%%%%%%%%%%%%%%%%%%%%%%%%%%%%%%%%%%%%%%%%%%%%%%%%%%%%%%%%%%%%%%%%%%%%%
% Es folgen einige Hilfsmittel, die du einbinden kannst, aber nicht mußt!
% Zum verwenden, entferne einfach in der entsprechenden Zeile das Prozentzeichen

%\pagewiselinenumbers				% Seitenweise Zeilennummerierung einschalten
%\onehalfspacing					% aktiviert 1.5-fachen Zeilenabstand
%\doublespacing						% aktiviert doppelten Zeilenabstand

%%%%%%%%%%%%%%%%%%%%%%%%%%%%%%%%%%%%%%%%%%%%%%%%%%%%%%%%%%%%%%%%%%%%%%%%%%%%%%
\tableofcontents					% Inhaltsverzeichnis
%%%%%%%%%%%%%%%%%%%%%%%%%%%%%%%%%%%%%%%%%%%%%%%%%%%%%%%%%%%%%%%%%%%%%%%%%%%%%%
% Aus Datenschutzgründen wird der Text Ihrer Abschlussarbeit in pseudonymisierter Form zur Plagiatsprüfung weitergeleitet. Eine etwaige Danksagung muss sich innerhalb der ersten 6 Seiten der Arbeit befinden, da diese vor der Übermittlung an die Texterkennungssoftware entfernt werden.

\chapter*{Danksagung}
\label{ch:danksagung}
%%%%%%%%%%%%%%%%%%%%%%%%%%%%%%%%%%%%%%%%%%%%%%%%%%%%%%%%%%%%%%%%%%%%%%%%%%%%%%
% Die Zusammenfassung/Der Abstract scheint nicht im Inhaltsverzeichnis auf.
%%%%%%%%%%%%%%%%%%%%%%%%%%%%%%%%%%%%%%%%%%%%%%%%%%%%%%%%%%%%%%%%%%%%%%%%%%%%%%
\chapter*{\abstractname}

%%%%%%%%%%%%%%%%%%%%%%%%%%%%%%%%%%%%%%%%%%%%%%%%%%%%%%%%%%%%%%%%%%%%%%%%%%%%%%

%%%%%%%%%%%%%%%%%%%%%%%%%%%%%%%%%%%%%%%%%%%%%%%%%%%%%%%%%%%%%%%%%%%%%%%%%%%%%%
% Hier beginnt der Haupttext.
%%%%%%%%%%%%%%%%%%%%%%%%%%%%%%%%%%%%%%%%%%%%%%%%%%%%%%%%%%%%%%%%%%%%%%%%%%%%%%

\chapter{Fragestellung}
\label{ch:fragestellung} % für einen Textverweis auf das Kapitel: \ref{ch:fragestellung}

\chapter{Einleitung}
\label{Chapter::Einleitung}

In der modernen Datenanalyse spielen gemischte Modelle eine zentrale Rolle, da sie es ermöglichen, sowohl feste als auch zufällige Effekte zu berücksichtigen, was sie besonders in den Bereichen der Biostatistik, der Sozialwissenschaften und der ökonomischen Modellierung beliebt macht. Mit dem Aufkommen von Big Data und komplexen Datenstrukturen hat sich der Fokus zunehmend auf die effiziente und genaue Extraktion von Informationen aus großen und oft unübersichtlichen Datensätzen verschoben. In diesem Kontext gewinnen latente Repräsentationen an Bedeutung, da sie es ermöglichen, die inhärenten Strukturen innerhalb der Daten zu identifizieren und zu nutzen, um tiefergehende Einsichten zu gewinnen.

Jedoch birgt die Integration von gemischten Modellen in latente Repräsentationen das Risiko einer Verzerrung der Inferenzergebnisse, was die Genauigkeit und Zuverlässigkeit der aus den Daten gezogenen Schlussfolgerungen erheblich beeinträchtigen kann. Diese Arbeit beschäftigt sich daher mit der Untersuchung der Verzerrungen, die bei der Anwendung gemischter Modelle auf latente Repräsentationen auftreten können. Ziel ist es, die Mechanismen zu verstehen, die zu diesen Verzerrungen führen, und Methoden zu entwickeln, um ihre Auswirkungen zu minimieren.

Die Fragestellung der Verzerrung ist besonders relevant, da eine fehlerhafte Inferenz zu falschen Entscheidungen führen kann, die in praktischen Anwendungen schwerwiegende Folgen haben könnten. Durch eine sorgfältige Analyse und Evaluation der gemischten Modellansätze in Verbindung mit latenten Repräsentationen strebt diese Arbeit an, einen Beitrag zur Verbesserung der Modellgenauigkeit und der Verlässlichkeit von Inferenzschlüssen zu leisten.

Diese Arbeit gliedert sich in mehrere Teile, die zunächst die theoretischen Grundlagen der gemischten Modelle und der latenten Repräsentationen behandeln, gefolgt von einer Diskussion der Methoden zur Messung und Korrektur von Verzerrungen. Anhand von experimentellen Studien werden diese Konzepte dann praktisch angewendet und evaluiert, um abschließend Empfehlungen für die Anwendung dieser Techniken in der Forschung und Praxis zu geben.
\section{Motivation}
\label{Section::Motivation}
  
\section{Problem and Research Questions}
\label{Section::Problem_and_Research_Questions}

%\ifstandalone
%    \printglossary
%    \printbibliography[heading=bibintoc]
%\fi

\end{document}

%%%%%%%%%%%%%%%%%%%%%%%%%%%%%%%%%%%%%%%%%%%%%%%%%%%%%%%%%%%%%%%%%%%%%%%%%%%%%%
\chapter{Methodik}
\label{ch:methodik}	% für einen Textverweis auf das Kapitel: \ref{ch:methodik}


%%%%%%%%%%%%%%%%%%%%%%%%%%%%%%%%%%%%%%%%%%%%%%%%%%%%%%%%%%%%%%%%%%%%%%%%%%%%%%
\chapter{Ergebnisse}
\label{ch:ergebnisse}


%%%%%%%%%%%%%%%%%%%%%%%%%%%%%%%%%%%%%%%%%%%%%%%%%%%%%%%%%%%%%%%%%%%%%%%%%%%%%%
\chapter{Diskussion}
\label{ch:diskussion}

%%%%%%%%%%%%%%%%%%%%%%%%%%%%%%%%%%%%%%%%%%%%%%%%%%%%%%%%%%%%%%%%%%%%%%%%%%%%%%
\chapter{Schlussfolgerung}
\label{ch:schlussfolgerung}


%%%%%%%%%%%%%%%%%%%%%%%%%%%%%%%%%%%%%%%%%%%%%%%%%%%%%%%%%%%%%%%%%%%%%%%%%%%%%%

\listoffigures   	% Verzeichnis der Abbildungen
\listoftables     	% Verzeichnis der Tabellen

%\nocite{*} 		% ALLE Zitate aus der Literaturdatenbank werden übernommen!

% Der Stil der Literaturzitate. Folgende Möglichkeiten sind günstig.
%  	apacite		= Author(Datum) .. komplett englisch
%	gerapalike	= Author(Datum) .. komplett deutsch
%  	gerplain	= [1] 			.. komplett deutsch
%	apalike2	= Author(Datum) .. teilweise für deutsch geeignet
%  	apa 		= Author(Datum) .. komplett englisch
%	elsart-harv	= Author(Datum) .. komplett englisch
%	chicago		= Author(Datum) .. komplett englisch
%	plain		= [1]			.. komplett englisch
%	nature		= ²				.. komplett englisch

\bibliographystyle{apacite} 
\bibliography{literature}


%%%%%%%%%%%%%%%%%%%%%%%%%%%%%%%%%%%%%%%%%%%%%%%%%%%%%%%%%%%%%%%%%%%%%%%%%%%%%%

\end{document}						% Ende des Dokuments
