\documentclass[%
thesis=student,% bachlor's or master's thesis
coverpage=false,% do not print an extra cover page
titlepage=false,% do not print an extra title page
headmarks=true, % headmarks can be switched on or off
english,% or `german`
font=libertine, % use `libertine` font; alternatives: `helvet` / `palatino` / `times`
math=newpxtx, % math font `newpxtx`; alternatives: `ams`, `pxtx`
BCOR=5mm,% binding correction - adapt accordingly
coverBCOR=11mm% binding correction for the cover - adapt accordingly
]{tumbook}

\makeatletter %redefine some labels from the TUM template
\provideName{\@tum@examiner@}{Supervisor}{Themensteller} % or `Themenstellerin`
\provideName{\@tum@supervisor@}{Advisors}{Betreuer} % or `Advisor` / `Betreuerin`
\makeatother

\usepackage{booktabs}% for more beautiful tables
\usepackage{cleveref}% intelligent references

\usepackage{diffcoeff}
\usepackage{amsmath}
\newtheorem{definition}{Definition}[section]
\newtheorem{theorem}[definition]{Satz}

%Literatur
\usepackage[%
    backend=bibtex, %, or `biber` on more up-to-date systems
    sortcites, % sort automatically
    sorting=nty, % sort order
    safeinputenc, % solves problems with unicode-formatted author names etc.
    citestyle=alphabetic, %
    bibstyle=alphabetic, %
    hyperref=true, % provide clickable links
    maxbibnames=3, % shorten author list for more than 3 names
    maxcitenames=3, % use at most 3 names for key
    url=false, % do not print URLs
    doi=false, % do not print DOIs
    giveninits=true,
    ]%
{biblatex}
\addbibresource{literature.bib}

% automatische Anführungszeichen
\usepackage[autostyle=true]{csquotes}


\title{How to Write an Excellent Mathematical Thesis}
\subtitle{A Comprehensive Guide for TUM Students}

\author{Felix Klein}

\degree{Master of Science}% or `Bachlor of Science`
\dateSubmitted{31. März 2023}% preferably use some universally recognized date format


\examiner{Prof.\@ Dr.\@ rer.\@ nat.\@ Adolf Hurwitz}% `Themensteller`
\supervisor{Dr.\@ L. Euler\\Dr.\@ C.-F. Gauß}% `Betreuer`


\begin{document}

\frontmatter
\maketitle

\section*{Zusammenfassung}
Eine kurze Zusammenfassung der Arbeit auf Deutsch.

\section*{Abstract}
A brief abstract of this thesis in English.

\cleardoublepage{}

\tableofcontents

\mainmatter{}
\chapter{Introduction}

To use the \LaTeX{} templates provided here you will need to add the directory \verb|tum-templates| as a local package directory to your \LaTeX{} distribution. An easy way to do this is by setting the environment variable \verb|TEXINPUTS| to \verb|.//:| on Linux/Mac systems and to \verb|.//;| on a windows machine (meaning: search the current directory and its subdirectories for packages first, then use the usual search path). On a Linux or Mac you can compile this document to a PDF file in a terminal through the following commands (the first command needs to be issued only once):
\begin{verbatim}
export TEXINPUTS=.//:
pdflatex master
bibtex master
pdflatex master
\end{verbatim}

On a windows computer, you would use the following commands in a terminal:

\begin{verbatim}
set TEXINPUTS=.//;
pdflatex master
bibtex master
pdflatex master
\end{verbatim}


\section{First Section of the Introduction}%
\label{sec:first-sect-intr}
Hier folgt eine ausführliche Erklärung und Motivation. Insbesondere weisen wir auf den wunderbaren Artikel von \textcite{Edmonds:1965} und auf~\cite{GareyJohnson:1979} für weitere Hintergründe.

\section{Second Section of the Introduction}%
\label{sec:second-sect-intr}

Wichtige Informationen finden sich in \cref{tab:wonderful-table}.

\begin{table}[hbt]
  \centering
  \begin{tabular}{rl}
    \toprule%
    \textbf{Name}& \textbf{Place of Birth}\\ \midrule
    Gauß & Braunschweig\\
    Euler & Basel\\
    Edmonds & Washington, D.\@C.\@\\
    \bottomrule
  \end{tabular}

  \caption{A most wonderful table}%
  \label{tab:wonderful-table}
\end{table}

\subsection{A Lonesome Subsection}%
\label{sec:lonesome-subsection}
Eine ausführliche \enquote{Erklärung} findet der aufmerksame Leser in \cref{sec:first-sect-intr}.

\clearpage{}

Hier geht es weiter mit dem Text.

\chapter{Mathematical Foundations}

\section{Definitions}
\begin{definition}[Definitheit]
  Hier definieren wir definitive Definitheit.
\end{definition}

\begin{theorem}[vom X]
  War wohl nix. Es gilt aber
  \begin{align*}
    \sum_{i=1}^{n} f_i(x) = \int \hat{f}(x) \dl{x}
  \end{align*}
\end{theorem}

\appendix
\chapter{Appendix}
\section{Supporting Data}
\section{Some Code Listings}

\backmatter{}
\listoffigures% may be removed
\listoftables% may be removed

\nocite{Alspach:2008,GaleShapley:1962} % further literature that has not been explicitly referenced in the text
\printbibliography{} % print bibliography

\end{document}

%%% Local Variables:
%%% mode: latex
%%% TeX-engine: default
%%% TeX-command-extra-options: "-shell-escape"
%%% ispell-local-dictionary: "american"
%%% eval: (setenv "TEXINPUTS" ".//:")
%%% TeX-master: t
%%% End:
